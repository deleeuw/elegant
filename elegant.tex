% Options for packages loaded elsewhere
% Options for packages loaded elsewhere
\PassOptionsToPackage{unicode}{hyperref}
\PassOptionsToPackage{hyphens}{url}
\PassOptionsToPackage{dvipsnames,svgnames,x11names}{xcolor}
%
\documentclass[
  12pt,
  letterpaper,
  DIV=11,
  numbers=noendperiod]{scrartcl}
\usepackage{xcolor}
\usepackage{amsmath,amssymb}
\setcounter{secnumdepth}{5}
\usepackage{iftex}
\ifPDFTeX
  \usepackage[T1]{fontenc}
  \usepackage[utf8]{inputenc}
  \usepackage{textcomp} % provide euro and other symbols
\else % if luatex or xetex
  \usepackage{unicode-math} % this also loads fontspec
  \defaultfontfeatures{Scale=MatchLowercase}
  \defaultfontfeatures[\rmfamily]{Ligatures=TeX,Scale=1}
\fi
\usepackage{lmodern}
\ifPDFTeX\else
  % xetex/luatex font selection
  \setmainfont[]{Times New Roman}
\fi
% Use upquote if available, for straight quotes in verbatim environments
\IfFileExists{upquote.sty}{\usepackage{upquote}}{}
\IfFileExists{microtype.sty}{% use microtype if available
  \usepackage[]{microtype}
  \UseMicrotypeSet[protrusion]{basicmath} % disable protrusion for tt fonts
}{}
\makeatletter
\@ifundefined{KOMAClassName}{% if non-KOMA class
  \IfFileExists{parskip.sty}{%
    \usepackage{parskip}
  }{% else
    \setlength{\parindent}{0pt}
    \setlength{\parskip}{6pt plus 2pt minus 1pt}}
}{% if KOMA class
  \KOMAoptions{parskip=half}}
\makeatother
% Make \paragraph and \subparagraph free-standing
\makeatletter
\ifx\paragraph\undefined\else
  \let\oldparagraph\paragraph
  \renewcommand{\paragraph}{
    \@ifstar
      \xxxParagraphStar
      \xxxParagraphNoStar
  }
  \newcommand{\xxxParagraphStar}[1]{\oldparagraph*{#1}\mbox{}}
  \newcommand{\xxxParagraphNoStar}[1]{\oldparagraph{#1}\mbox{}}
\fi
\ifx\subparagraph\undefined\else
  \let\oldsubparagraph\subparagraph
  \renewcommand{\subparagraph}{
    \@ifstar
      \xxxSubParagraphStar
      \xxxSubParagraphNoStar
  }
  \newcommand{\xxxSubParagraphStar}[1]{\oldsubparagraph*{#1}\mbox{}}
  \newcommand{\xxxSubParagraphNoStar}[1]{\oldsubparagraph{#1}\mbox{}}
\fi
\makeatother


\usepackage{longtable,booktabs,array}
\usepackage{calc} % for calculating minipage widths
% Correct order of tables after \paragraph or \subparagraph
\usepackage{etoolbox}
\makeatletter
\patchcmd\longtable{\par}{\if@noskipsec\mbox{}\fi\par}{}{}
\makeatother
% Allow footnotes in longtable head/foot
\IfFileExists{footnotehyper.sty}{\usepackage{footnotehyper}}{\usepackage{footnote}}
\makesavenoteenv{longtable}
\usepackage{graphicx}
\makeatletter
\newsavebox\pandoc@box
\newcommand*\pandocbounded[1]{% scales image to fit in text height/width
  \sbox\pandoc@box{#1}%
  \Gscale@div\@tempa{\textheight}{\dimexpr\ht\pandoc@box+\dp\pandoc@box\relax}%
  \Gscale@div\@tempb{\linewidth}{\wd\pandoc@box}%
  \ifdim\@tempb\p@<\@tempa\p@\let\@tempa\@tempb\fi% select the smaller of both
  \ifdim\@tempa\p@<\p@\scalebox{\@tempa}{\usebox\pandoc@box}%
  \else\usebox{\pandoc@box}%
  \fi%
}
% Set default figure placement to htbp
\def\fps@figure{htbp}
\makeatother


% definitions for citeproc citations
\NewDocumentCommand\citeproctext{}{}
\NewDocumentCommand\citeproc{mm}{%
  \begingroup\def\citeproctext{#2}\cite{#1}\endgroup}
\makeatletter
 % allow citations to break across lines
 \let\@cite@ofmt\@firstofone
 % avoid brackets around text for \cite:
 \def\@biblabel#1{}
 \def\@cite#1#2{{#1\if@tempswa , #2\fi}}
\makeatother
\newlength{\cslhangindent}
\setlength{\cslhangindent}{1.5em}
\newlength{\csllabelwidth}
\setlength{\csllabelwidth}{3em}
\newenvironment{CSLReferences}[2] % #1 hanging-indent, #2 entry-spacing
 {\begin{list}{}{%
  \setlength{\itemindent}{0pt}
  \setlength{\leftmargin}{0pt}
  \setlength{\parsep}{0pt}
  % turn on hanging indent if param 1 is 1
  \ifodd #1
   \setlength{\leftmargin}{\cslhangindent}
   \setlength{\itemindent}{-1\cslhangindent}
  \fi
  % set entry spacing
  \setlength{\itemsep}{#2\baselineskip}}}
 {\end{list}}
\usepackage{calc}
\newcommand{\CSLBlock}[1]{\hfill\break\parbox[t]{\linewidth}{\strut\ignorespaces#1\strut}}
\newcommand{\CSLLeftMargin}[1]{\parbox[t]{\csllabelwidth}{\strut#1\strut}}
\newcommand{\CSLRightInline}[1]{\parbox[t]{\linewidth - \csllabelwidth}{\strut#1\strut}}
\newcommand{\CSLIndent}[1]{\hspace{\cslhangindent}#1}



\setlength{\emergencystretch}{3em} % prevent overfull lines

\providecommand{\tightlist}{%
  \setlength{\itemsep}{0pt}\setlength{\parskip}{0pt}}



 


\usepackage{tcolorbox}
\usepackage{amssymb}
\usepackage{yfonts}
\usepackage{bm}


\newtcolorbox{greybox}{
  colback=white,
  colframe=blue,
  coltext=black,
  boxsep=5pt,
  arc=4pt}
  
\newcommand{\sectionbreak}{\clearpage}

 
\newcommand{\ds}[4]{\sum_{{#1}=1}^{#3}\sum_{{#2}=1}^{#4}}
\newcommand{\us}[3]{\mathop{\sum\sum}_{1\leq{#2}<{#1}\leq{#3}}}

\newcommand{\ol}[1]{\overline{#1}}
\newcommand{\ul}[1]{\underline{#1}}

\newcommand{\amin}[1]{\mathop{\text{argmin}}_{#1}}
\newcommand{\amax}[1]{\mathop{\text{argmax}}_{#1}}

\newcommand{\ci}{\perp\!\!\!\perp}

\newcommand{\mc}[1]{\mathcal{#1}}
\newcommand{\mb}[1]{\mathbb{#1}}
\newcommand{\mf}[1]{\mathfrak{#1}}

\newcommand{\eps}{\epsilon}
\newcommand{\lbd}{\lambda}
\newcommand{\alp}{\alpha}
\newcommand{\df}{=:}
\newcommand{\am}[1]{\mathop{\text{argmin}}_{#1}}
\newcommand{\ls}[2]{\mathop{\sum\sum}_{#1}^{#2}}
\newcommand{\ijs}{\mathop{\sum\sum}_{1\leq i<j\leq n}}
\newcommand{\jis}{\mathop{\sum\sum}_{1\leq j<i\leq n}}
\newcommand{\sij}{\sum_{i=1}^n\sum_{j=1}^n}
	
\KOMAoption{captions}{tableheading}
\makeatletter
\@ifpackageloaded{caption}{}{\usepackage{caption}}
\AtBeginDocument{%
\ifdefined\contentsname
  \renewcommand*\contentsname{Table of contents}
\else
  \newcommand\contentsname{Table of contents}
\fi
\ifdefined\listfigurename
  \renewcommand*\listfigurename{List of Figures}
\else
  \newcommand\listfigurename{List of Figures}
\fi
\ifdefined\listtablename
  \renewcommand*\listtablename{List of Tables}
\else
  \newcommand\listtablename{List of Tables}
\fi
\ifdefined\figurename
  \renewcommand*\figurename{Figure}
\else
  \newcommand\figurename{Figure}
\fi
\ifdefined\tablename
  \renewcommand*\tablename{Table}
\else
  \newcommand\tablename{Table}
\fi
}
\@ifpackageloaded{float}{}{\usepackage{float}}
\floatstyle{ruled}
\@ifundefined{c@chapter}{\newfloat{codelisting}{h}{lop}}{\newfloat{codelisting}{h}{lop}[chapter]}
\floatname{codelisting}{Listing}
\newcommand*\listoflistings{\listof{codelisting}{List of Listings}}
\makeatother
\makeatletter
\makeatother
\makeatletter
\@ifpackageloaded{caption}{}{\usepackage{caption}}
\@ifpackageloaded{subcaption}{}{\usepackage{subcaption}}
\makeatother
\usepackage{bookmark}
\IfFileExists{xurl.sty}{\usepackage{xurl}}{} % add URL line breaks if available
\urlstyle{same}
\hypersetup{
  pdftitle={Squared Distance Scaling},
  pdfauthor={Jan de Leeuw},
  colorlinks=true,
  linkcolor={blue},
  filecolor={Maroon},
  citecolor={Blue},
  urlcolor={Blue},
  pdfcreator={LaTeX via pandoc}}


\title{Squared Distance Scaling}
\author{Jan de Leeuw}
\date{June 9, 2025}
\begin{document}
\maketitle
\begin{abstract}
TBD
\end{abstract}

\renewcommand*\contentsname{Table of contents}
{
\hypersetup{linkcolor=}
\setcounter{tocdepth}{3}
\tableofcontents
}

\textbf{Note:} This is a working manuscript which will be
expanded/updated frequently. All suggestions for improvement are
welcome. All Rmd, tex, html, pdf, R, and C files are in the public
domain. Attribution will be appreciated, but is not required. The files
can be found at \url{https://github.com/deleeuw/elegant}

\sectionbreak

\section{Introduction}\label{introduction}

In \emph{squared distance multidimensional scaling} we minimize the
least squares loss function \begin{equation}
\sigma(X):=\sum_{k=1}^m w_k(\delta_k^2-d_k^2(X))^2\label{eq-sstress}
\end{equation} over \(n\times p\) configurations \(X\). Here the
\(\delta_k\) are known non-negative \emph{pseudo-distances}, the \(w_k\)
are known positive \emph{weights}, and the \(d_k(X)\) are Euclidean
distances. Each index \(k\) in \eqref{eq-sstress} corresponds with a
pair of indices \((i,j)\), with both \(1\leq i\leq n\) and
\(1\leq j\leq n\). Thus we try to find a configuration of \(n\) points
on \(\mathbb{R}^p\) such that the distances between the points
approximate the corresponding pseudo-distances in the data.

Loss function \eqref{eq-sstress} is traditionally known as
\emph{sstress}. In the \emph{ALSCAL} program for squared distance
scaling (Takane, Young, and De Leeuw
(\citeproc{ref-takane_young_deleeuw_A_77}{1977})) a coordinate descent
algorithm is proposed to minimize loss, in which each iteration cycle
consists of minimizing \(np\) univariate quartics. There have been quite
a few alternative algorithms proposed, both in the area of
multidimensional scaling (De Leeuw (\citeproc{ref-deleeuw_U_75b}{1975}),
Browne (\citeproc{ref-browne_87}{1987}), Kearsley, Tapia, and Trosset
(\citeproc{ref-kearsley_tapia_trosset_94}{1994})) and in that of
low-rank distance matrix completion (Mishra, Meyer, and Sepulchre
(\citeproc{ref-mishra_meyer_sepulchre_11}{2011})).

The reference section of this paper does not have publication
information on De Leeuw (\citeproc{ref-deleeuw_U_75b}{1975}), in fact
not even a URL, because the paper somehow got lost in the folds of time
(Takane (\citeproc{ref-takane_16}{2016})). Proof of its existence are
references to it in Takane (\citeproc{ref-takane_77}{1977}) and Browne
(\citeproc{ref-browne_87}{1987}). At the time it was concluded that the
algorithm proposed in De Leeuw (\citeproc{ref-deleeuw_U_75b}{1975}),
which was proudly baptized as \emph{ELEGANT}, converged too slowly to be
practical. Recent attempts to revive and improve it are De Leeuw,
Groenen, and Pietersz
(\citeproc{ref-deleeuw_groenen_pietersz_E_16m}{2016}) and De Leeuw
(\citeproc{ref-deleeuw_E_16o}{2016}). This paper is another such
attempt.

\sectionbreak

\section{Majorization}\label{majorization}

The original derivation of the algorithm in De Leeuw
(\citeproc{ref-deleeuw_U_75b}{1975}) was based on \emph{augmentation}.
The derivation is reviewed in De Leeuw, Groenen, and Pietersz
(\citeproc{ref-deleeuw_groenen_pietersz_E_16m}{2016}). But improvements
are possible if we discuss it in the general framework of majorization
(De Leeuw (\citeproc{ref-deleeuw_C_96}{1996})), currently more
familiarly known as MM ((\citeproc{ref-lange_}{\textbf{lange\_?}})).

We change variables from \(X\) to \(C=XX'\). Thus \[
\sigma(C)=\sum_{k=1}^m w_k(\delta_k^2-\text{tr}\ A_kC)^2
\] Define \[
B:=\sum_{k=1}^m w_k\delta_k^2A_k,
\] and \[
V:=\sum_{k=1}^m w_ka_ka_k'
\] with \(a_k=\text{vec}(A_k)\). Then \[
\sigma(c)=K-2b'c+c'Vc,
\] with \(c=\text{vec}(C)\) and \(b=\text{vec}(B)\).

To start the majorization, use \(c=\tilde c+(c-\tilde c)\). Then \[
\sigma(c)=\sigma(\tilde c)-2(c-\tilde c)'(b-V\tilde c)+(c-\tilde c)'V(c-\tilde c),
\] and thus \[
\sigma(c)\leq\sigma(\tilde c)-2(c-\tilde c)'(b-V\tilde c)+\lambda(c-\tilde c)'(c-\tilde c)=
\sigma(\tilde c)+\lambda((c-\tilde c)-g)'((c-\tilde c)-g)-\lambda g'g,
\] with \(\lambda\) the largest eigenvalue of \(V\) and
\(g:=\lambda^{-1}(b-V\tilde c)\).

In a majozation step we minimize \((c-\overline c)'(c-\overline c),\)
where \[\overline{c}:=\tilde c+\lambda^{-1}(b-V\tilde c)\].

Taking the inverse of vec show that we must minimize \[
\text{tr}\ (\overline{C}-XX')^2
\] over \(X\).

\sectionbreak

\section{Details}\label{details}

Computing \(\lambda\) is simplified by noting that the largest
eigenvalue of \(V\) is equal to the largest eigenvalue of
\(W^\frac12HW^\frac12\), where \(H\) has elements \[
h_{kl}=a_k'a_l^{\ }.
\] The elements of \(H\) are non-negative. Also \(h_{kl}\) is equal to 4
if \(k=l\) and equal to 1 if \(A_k\) and \(A_l\) have one index in
common, otherwise it is zero. It follows that in the complete case, with
\(m=n(n-1)/2\), and in addition, if there are unit weights,
\(\lambda=2n\). In the incomplete case, still with unit weights,
\(\lambda\leq 2n\).

\[
V\tilde c=\sum_{k=1}^m w_ka_ka_k'\tilde c=\sum_{k=1}^m w_ka_k\text{tr}\ A_k\tilde C=\sum_{k=1}^m w_ka_kd_k^2(\tilde C)
\] and thus \[
\text{vec}^{-1}(B-V\tilde c)=\sum_{k=1}^m w_k(\delta_k^2-d_k^2(\tilde C)A_k
\]

If it is too expensive to calculate the largest eigenvalue, we can use
the bound \[
\lambda=\max_x
\frac{x'W^\frac12HW^\frac12x}{x'x}=
\max_x\frac{x'W^\frac12HW^\frac12x}{x'Wx}\frac{x'Wx}{x'x}\leq 2n\max_k w_k
\] This is a major improvement ofthe bound that is us3ed, either
explicitly or implicitly, in
(\citeproc{ref-deleeuw_75}{\textbf{deleeuw\_75?}}) and
(\citeproc{ref-deleeuw_groenen_pietersz}{\textbf{deleeuw\_groenen\_pietersz?}}),
which is \[
\lambda\leq\text{tr}\ W^\frac12HW^\frac12 = 4\sum_{k-1}^m w_k.
\]

\sectionbreak

\section*{References}\label{references}
\addcontentsline{toc}{section}{References}

\phantomsection\label{refs}
\begin{CSLReferences}{1}{0}
\bibitem[\citeproctext]{ref-browne_87}
Browne, M. W. 1987. {``{The Young-Householder Algorithm and the Least
Squares Multidimensional Scaling of Squared Distances}.''} \emph{Journal
of Classification} 4: 175--90.

\bibitem[\citeproctext]{ref-deleeuw_U_75b}
De Leeuw, J. 1975. {``{An Alternating Least Squares Approach to Squared
Distance Scaling}.''} Department of Data Theory FSW/RUL.

\bibitem[\citeproctext]{ref-deleeuw_C_96}
---------. 1996. {``{Series Editors Introduction}.''} In \emph{{T.
Heinen: Latent Class and Discrete Latent Trait Models}}. Advanced
Quantitative Techniques in the Social Sciences 6. Thousand Oaks, CA:
Sage Publications.

\bibitem[\citeproctext]{ref-deleeuw_E_16o}
---------. 2016. {``Convergence Rate of {ELEGANT} Algorithms.''} 2016.
\url{https://jansweb.netlify.app/publication/deleeuw-e-16-o/deleeuw-e-16-o.pdf}.

\bibitem[\citeproctext]{ref-deleeuw_groenen_pietersz_E_16m}
De Leeuw, J., P. Groenen, and R. Pietersz. 2016. {``An Alternating Least
Squares Approach to Squared Distance Scaling.''}
\url{https://jansweb.netlify.app/publication/deleeuw-groenen-pietersz-e-16-m/deleeuw-groenen-pietersz-e-16-m.pdf}.

\bibitem[\citeproctext]{ref-kearsley_tapia_trosset_94}
Kearsley, A. J., R. A. Tapia, and M. W. Trosset. 1994. {``{The Solution
of the Metric STRESS and SSTRESS Problems in Multidimensional Scaling
Using Newton's Method}.''} TR94-44. Houston, TX: {Department of
Computational and Applied Mathematics, Rice University}.
\url{https://apps.dtic.mil/sti/pdfs/ADA445621.pdf}.

\bibitem[\citeproctext]{ref-mishra_meyer_sepulchre_11}
Mishra, B., G. Meyer, and R. Sepulchre. 2011. {``Low-Rank Optimization
for Distance Matrix Completion.''} In \emph{50th IEEE Conference on
Decision and Control and European Control Conference, Orlando, FL, USA,
2011}, 4455--60. \url{https://arxiv.org/pdf/1304.6663}.

\bibitem[\citeproctext]{ref-takane_77}
Takane, Y. 1977. {``{On the Relations among Four Methods of
Multidimensional Scaling}.''} \emph{Behaviormetrika} 4: 29--42.

\bibitem[\citeproctext]{ref-takane_16}
---------. 2016. {``{My Early Interactions with Jan and Some of His Lost
Papers}.''} \emph{Journal of Statistical Software} 73 (7): 1--14.
\url{https://www.jstatsoft.org/article/view/v073i07}.

\bibitem[\citeproctext]{ref-takane_young_deleeuw_A_77}
Takane, Y., F. W. Young, and J. De Leeuw. 1977. {``Nonmetric Individual
Differences in Multidimensional Scaling: An Alternating Least Squares
Method with Optimal Scaling Features.''} \emph{Psychometrika} 42: 7--67.

\end{CSLReferences}




\end{document}
