% Options for packages loaded elsewhere
% Options for packages loaded elsewhere
\PassOptionsToPackage{unicode}{hyperref}
\PassOptionsToPackage{hyphens}{url}
\PassOptionsToPackage{dvipsnames,svgnames,x11names}{xcolor}
%
\documentclass[
  12pt,
  letterpaper,
  DIV=11,
  numbers=noendperiod]{scrartcl}
\usepackage{xcolor}
\usepackage{amsmath,amssymb}
\setcounter{secnumdepth}{5}
\usepackage{iftex}
\ifPDFTeX
  \usepackage[T1]{fontenc}
  \usepackage[utf8]{inputenc}
  \usepackage{textcomp} % provide euro and other symbols
\else % if luatex or xetex
  \usepackage{unicode-math} % this also loads fontspec
  \defaultfontfeatures{Scale=MatchLowercase}
  \defaultfontfeatures[\rmfamily]{Ligatures=TeX,Scale=1}
\fi
\usepackage{lmodern}
\ifPDFTeX\else
  % xetex/luatex font selection
  \setmainfont[]{Times New Roman}
\fi
% Use upquote if available, for straight quotes in verbatim environments
\IfFileExists{upquote.sty}{\usepackage{upquote}}{}
\IfFileExists{microtype.sty}{% use microtype if available
  \usepackage[]{microtype}
  \UseMicrotypeSet[protrusion]{basicmath} % disable protrusion for tt fonts
}{}
\makeatletter
\@ifundefined{KOMAClassName}{% if non-KOMA class
  \IfFileExists{parskip.sty}{%
    \usepackage{parskip}
  }{% else
    \setlength{\parindent}{0pt}
    \setlength{\parskip}{6pt plus 2pt minus 1pt}}
}{% if KOMA class
  \KOMAoptions{parskip=half}}
\makeatother
% Make \paragraph and \subparagraph free-standing
\makeatletter
\ifx\paragraph\undefined\else
  \let\oldparagraph\paragraph
  \renewcommand{\paragraph}{
    \@ifstar
      \xxxParagraphStar
      \xxxParagraphNoStar
  }
  \newcommand{\xxxParagraphStar}[1]{\oldparagraph*{#1}\mbox{}}
  \newcommand{\xxxParagraphNoStar}[1]{\oldparagraph{#1}\mbox{}}
\fi
\ifx\subparagraph\undefined\else
  \let\oldsubparagraph\subparagraph
  \renewcommand{\subparagraph}{
    \@ifstar
      \xxxSubParagraphStar
      \xxxSubParagraphNoStar
  }
  \newcommand{\xxxSubParagraphStar}[1]{\oldsubparagraph*{#1}\mbox{}}
  \newcommand{\xxxSubParagraphNoStar}[1]{\oldsubparagraph{#1}\mbox{}}
\fi
\makeatother

\usepackage{color}
\usepackage{fancyvrb}
\newcommand{\VerbBar}{|}
\newcommand{\VERB}{\Verb[commandchars=\\\{\}]}
\DefineVerbatimEnvironment{Highlighting}{Verbatim}{commandchars=\\\{\}}
% Add ',fontsize=\small' for more characters per line
\usepackage{framed}
\definecolor{shadecolor}{RGB}{241,243,245}
\newenvironment{Shaded}{\begin{snugshade}}{\end{snugshade}}
\newcommand{\AlertTok}[1]{\textcolor[rgb]{0.68,0.00,0.00}{#1}}
\newcommand{\AnnotationTok}[1]{\textcolor[rgb]{0.37,0.37,0.37}{#1}}
\newcommand{\AttributeTok}[1]{\textcolor[rgb]{0.40,0.45,0.13}{#1}}
\newcommand{\BaseNTok}[1]{\textcolor[rgb]{0.68,0.00,0.00}{#1}}
\newcommand{\BuiltInTok}[1]{\textcolor[rgb]{0.00,0.23,0.31}{#1}}
\newcommand{\CharTok}[1]{\textcolor[rgb]{0.13,0.47,0.30}{#1}}
\newcommand{\CommentTok}[1]{\textcolor[rgb]{0.37,0.37,0.37}{#1}}
\newcommand{\CommentVarTok}[1]{\textcolor[rgb]{0.37,0.37,0.37}{\textit{#1}}}
\newcommand{\ConstantTok}[1]{\textcolor[rgb]{0.56,0.35,0.01}{#1}}
\newcommand{\ControlFlowTok}[1]{\textcolor[rgb]{0.00,0.23,0.31}{\textbf{#1}}}
\newcommand{\DataTypeTok}[1]{\textcolor[rgb]{0.68,0.00,0.00}{#1}}
\newcommand{\DecValTok}[1]{\textcolor[rgb]{0.68,0.00,0.00}{#1}}
\newcommand{\DocumentationTok}[1]{\textcolor[rgb]{0.37,0.37,0.37}{\textit{#1}}}
\newcommand{\ErrorTok}[1]{\textcolor[rgb]{0.68,0.00,0.00}{#1}}
\newcommand{\ExtensionTok}[1]{\textcolor[rgb]{0.00,0.23,0.31}{#1}}
\newcommand{\FloatTok}[1]{\textcolor[rgb]{0.68,0.00,0.00}{#1}}
\newcommand{\FunctionTok}[1]{\textcolor[rgb]{0.28,0.35,0.67}{#1}}
\newcommand{\ImportTok}[1]{\textcolor[rgb]{0.00,0.46,0.62}{#1}}
\newcommand{\InformationTok}[1]{\textcolor[rgb]{0.37,0.37,0.37}{#1}}
\newcommand{\KeywordTok}[1]{\textcolor[rgb]{0.00,0.23,0.31}{\textbf{#1}}}
\newcommand{\NormalTok}[1]{\textcolor[rgb]{0.00,0.23,0.31}{#1}}
\newcommand{\OperatorTok}[1]{\textcolor[rgb]{0.37,0.37,0.37}{#1}}
\newcommand{\OtherTok}[1]{\textcolor[rgb]{0.00,0.23,0.31}{#1}}
\newcommand{\PreprocessorTok}[1]{\textcolor[rgb]{0.68,0.00,0.00}{#1}}
\newcommand{\RegionMarkerTok}[1]{\textcolor[rgb]{0.00,0.23,0.31}{#1}}
\newcommand{\SpecialCharTok}[1]{\textcolor[rgb]{0.37,0.37,0.37}{#1}}
\newcommand{\SpecialStringTok}[1]{\textcolor[rgb]{0.13,0.47,0.30}{#1}}
\newcommand{\StringTok}[1]{\textcolor[rgb]{0.13,0.47,0.30}{#1}}
\newcommand{\VariableTok}[1]{\textcolor[rgb]{0.07,0.07,0.07}{#1}}
\newcommand{\VerbatimStringTok}[1]{\textcolor[rgb]{0.13,0.47,0.30}{#1}}
\newcommand{\WarningTok}[1]{\textcolor[rgb]{0.37,0.37,0.37}{\textit{#1}}}

\usepackage{longtable,booktabs,array}
\usepackage{calc} % for calculating minipage widths
% Correct order of tables after \paragraph or \subparagraph
\usepackage{etoolbox}
\makeatletter
\patchcmd\longtable{\par}{\if@noskipsec\mbox{}\fi\par}{}{}
\makeatother
% Allow footnotes in longtable head/foot
\IfFileExists{footnotehyper.sty}{\usepackage{footnotehyper}}{\usepackage{footnote}}
\makesavenoteenv{longtable}
\usepackage{graphicx}
\makeatletter
\newsavebox\pandoc@box
\newcommand*\pandocbounded[1]{% scales image to fit in text height/width
  \sbox\pandoc@box{#1}%
  \Gscale@div\@tempa{\textheight}{\dimexpr\ht\pandoc@box+\dp\pandoc@box\relax}%
  \Gscale@div\@tempb{\linewidth}{\wd\pandoc@box}%
  \ifdim\@tempb\p@<\@tempa\p@\let\@tempa\@tempb\fi% select the smaller of both
  \ifdim\@tempa\p@<\p@\scalebox{\@tempa}{\usebox\pandoc@box}%
  \else\usebox{\pandoc@box}%
  \fi%
}
% Set default figure placement to htbp
\def\fps@figure{htbp}
\makeatother


% definitions for citeproc citations
\NewDocumentCommand\citeproctext{}{}
\NewDocumentCommand\citeproc{mm}{%
  \begingroup\def\citeproctext{#2}\cite{#1}\endgroup}
\makeatletter
 % allow citations to break across lines
 \let\@cite@ofmt\@firstofone
 % avoid brackets around text for \cite:
 \def\@biblabel#1{}
 \def\@cite#1#2{{#1\if@tempswa , #2\fi}}
\makeatother
\newlength{\cslhangindent}
\setlength{\cslhangindent}{1.5em}
\newlength{\csllabelwidth}
\setlength{\csllabelwidth}{3em}
\newenvironment{CSLReferences}[2] % #1 hanging-indent, #2 entry-spacing
 {\begin{list}{}{%
  \setlength{\itemindent}{0pt}
  \setlength{\leftmargin}{0pt}
  \setlength{\parsep}{0pt}
  % turn on hanging indent if param 1 is 1
  \ifodd #1
   \setlength{\leftmargin}{\cslhangindent}
   \setlength{\itemindent}{-1\cslhangindent}
  \fi
  % set entry spacing
  \setlength{\itemsep}{#2\baselineskip}}}
 {\end{list}}
\usepackage{calc}
\newcommand{\CSLBlock}[1]{\hfill\break\parbox[t]{\linewidth}{\strut\ignorespaces#1\strut}}
\newcommand{\CSLLeftMargin}[1]{\parbox[t]{\csllabelwidth}{\strut#1\strut}}
\newcommand{\CSLRightInline}[1]{\parbox[t]{\linewidth - \csllabelwidth}{\strut#1\strut}}
\newcommand{\CSLIndent}[1]{\hspace{\cslhangindent}#1}



\setlength{\emergencystretch}{3em} % prevent overfull lines

\providecommand{\tightlist}{%
  \setlength{\itemsep}{0pt}\setlength{\parskip}{0pt}}



 


\usepackage{tcolorbox}
\usepackage{amssymb}
\usepackage{yfonts}
\usepackage{bm}


\newtcolorbox{greybox}{
  colback=white,
  colframe=blue,
  coltext=black,
  boxsep=5pt,
  arc=4pt}
  
\newcommand{\sectionbreak}{\clearpage}

 
\newcommand{\ds}[4]{\sum_{{#1}=1}^{#3}\sum_{{#2}=1}^{#4}}
\newcommand{\us}[3]{\mathop{\sum\sum}_{1\leq{#2}<{#1}\leq{#3}}}

\newcommand{\ol}[1]{\overline{#1}}
\newcommand{\ul}[1]{\underline{#1}}

\newcommand{\amin}[1]{\mathop{\text{argmin}}_{#1}}
\newcommand{\amax}[1]{\mathop{\text{argmax}}_{#1}}

\newcommand{\ci}{\perp\!\!\!\perp}

\newcommand{\mc}[1]{\mathcal{#1}}
\newcommand{\mb}[1]{\mathbb{#1}}
\newcommand{\mf}[1]{\mathfrak{#1}}

\newcommand{\eps}{\epsilon}
\newcommand{\lbd}{\lambda}
\newcommand{\alp}{\alpha}
\newcommand{\df}{=:}
\newcommand{\am}[1]{\mathop{\text{argmin}}_{#1}}
\newcommand{\ls}[2]{\mathop{\sum\sum}_{#1}^{#2}}
\newcommand{\ijs}{\mathop{\sum\sum}_{1\leq i<j\leq n}}
\newcommand{\jis}{\mathop{\sum\sum}_{1\leq j<i\leq n}}
\newcommand{\sij}{\sum_{i=1}^n\sum_{j=1}^n}
	
\KOMAoption{captions}{tableheading}
\makeatletter
\@ifpackageloaded{caption}{}{\usepackage{caption}}
\AtBeginDocument{%
\ifdefined\contentsname
  \renewcommand*\contentsname{Table of contents}
\else
  \newcommand\contentsname{Table of contents}
\fi
\ifdefined\listfigurename
  \renewcommand*\listfigurename{List of Figures}
\else
  \newcommand\listfigurename{List of Figures}
\fi
\ifdefined\listtablename
  \renewcommand*\listtablename{List of Tables}
\else
  \newcommand\listtablename{List of Tables}
\fi
\ifdefined\figurename
  \renewcommand*\figurename{Figure}
\else
  \newcommand\figurename{Figure}
\fi
\ifdefined\tablename
  \renewcommand*\tablename{Table}
\else
  \newcommand\tablename{Table}
\fi
}
\@ifpackageloaded{float}{}{\usepackage{float}}
\floatstyle{ruled}
\@ifundefined{c@chapter}{\newfloat{codelisting}{h}{lop}}{\newfloat{codelisting}{h}{lop}[chapter]}
\floatname{codelisting}{Listing}
\newcommand*\listoflistings{\listof{codelisting}{List of Listings}}
\makeatother
\makeatletter
\makeatother
\makeatletter
\@ifpackageloaded{caption}{}{\usepackage{caption}}
\@ifpackageloaded{subcaption}{}{\usepackage{subcaption}}
\makeatother
\usepackage{bookmark}
\IfFileExists{xurl.sty}{\usepackage{xurl}}{} % add URL line breaks if available
\urlstyle{same}
\hypersetup{
  pdftitle={Squared Distance Scaling},
  pdfauthor={Jan de Leeuw},
  colorlinks=true,
  linkcolor={blue},
  filecolor={Maroon},
  citecolor={Blue},
  urlcolor={Blue},
  pdfcreator={LaTeX via pandoc}}


\title{Squared Distance Scaling}
\author{Jan de Leeuw}
\date{June 9, 2025}
\begin{document}
\maketitle
\begin{abstract}
We discuss an improved version of the \emph{ELEGANT} algorithm for
metric squared distance scaling (or, equivalently, for low-rank distance
matrix completion) first introduced by De Leeuw
(\citeproc{ref-deleeuw_U_75b}{1975}).
\end{abstract}

\renewcommand*\contentsname{Table of contents}
{
\hypersetup{linkcolor=}
\setcounter{tocdepth}{3}
\tableofcontents
}

\textbf{Note:} This is a working manuscript which will be
expanded/updated frequently. All suggestions for improvement are
welcome. All Rmd, tex, html, pdf, R, and C files are in the public
domain. Attribution will be appreciated, but is not required. The files
can be found at \url{https://github.com/deleeuw/elegant}

\sectionbreak

\section{Introduction}\label{introduction}

In \emph{squared distance multidimensional scaling} we minimize the
least squares loss
function\footnote{The symbol $:=$ is used for defintions.}
\begin{equation}
\sigma(X):=\sum_{k=1}^m w_k(\delta_k^2-d_k^2(X))^2\label{eq-sstress}
\end{equation} over \(n\times p\) configurations \(X\). Here the
\(\delta_k\) are known non-negative \emph{pseudo-distances}, the \(w_k\)
are known positive \emph{weights}, and the \(d_k(X)\) are Euclidean
distances. Each index \(k\) in \eqref{eq-sstress} corresponds with a
pair of indices \((i,j)\), with both \(1\leq i\leq n\) and
\(1\leq j\leq n\). Thus we try to find a configuration of \(n\) points
on \(\mathbb{R}^p\) such that the distances between the points
approximate the corresponding pseudo-distances in the data.

Loss function \eqref{eq-sstress} is traditionally known as
\emph{sstress}. In the \emph{ALSCAL} program for squared distance
scaling (Takane, Young, and De Leeuw
(\citeproc{ref-takane_young_deleeuw_A_77}{1977})) a coordinate descent
algorithm, in which each iteration cycle consists of minimizing \(np\)
univariate quartics, is used to minimize loss. There have been quite a
few alternative algorithms proposed, both in multidimensional scaling
(De Leeuw (\citeproc{ref-deleeuw_U_75b}{1975}), Browne
(\citeproc{ref-browne_87}{1987}), Kearsley, Tapia, and Trosset
(\citeproc{ref-kearsley_tapia_trosset_94}{1994})) and in low-rank
distance matrix completion (Mishra, Meyer, and Sepulchre
(\citeproc{ref-mishra_meyer_sepulchre_11}{2011})).

The reference section of the present paper does not have publication
information on De Leeuw (\citeproc{ref-deleeuw_U_75b}{1975}), in fact
not even a URL, because that paper somehow got lost in the folds of time
(Takane (\citeproc{ref-takane_16}{2016})). Proof of its existence are
references to it in Takane (\citeproc{ref-takane_77}{1977}) and Browne
(\citeproc{ref-browne_87}{1987}). At the time it was concluded that the
algorithm proposed in De Leeuw (\citeproc{ref-deleeuw_U_75b}{1975}),
which was proudly baptized \emph{ELEGANT}, converged too slowly to be
practical. Recent attempts to revive and improve it are De Leeuw,
Groenen, and Pietersz
(\citeproc{ref-deleeuw_groenen_pietersz_E_16m}{2016}) and De Leeuw
(\citeproc{ref-deleeuw_E_16o}{2016}). This paper is another such
attempt.

\sectionbreak

\section{Majorization}\label{majorization}

The original derivation of the algorithm in De Leeuw
(\citeproc{ref-deleeuw_U_75b}{1975}) was based on \emph{augmentation}.
The derivation is reviewed in De Leeuw, Groenen, and Pietersz
(\citeproc{ref-deleeuw_groenen_pietersz_E_16m}{2016}). For a general
discussion of augmentation, see De Leeuw
(\citeproc{ref-deleeuw_C_94c}{1994}). Improvements of \emph{ELEGANT} are
possible if we discuss it in the general framework of majorization ,
currently more widely known as MM (De Leeuw
(\citeproc{ref-deleeuw_C_94c}{1994}), Heiser
(\citeproc{ref-heiser_95}{1995}), Lange
(\citeproc{ref-lange_16}{2016})).

We start by changing variables from \(X\) to \(C=XX'\). Thus
\begin{equation}
\sigma(C):=\sum_{k=1}^m w_k(\delta_k^2-\text{tr}\ A_kC)^2.\label{eq-csstress}
\end{equation} If \(k\) indexes pair \((i,j)\) then
\(A_k:=(e_i-e_j)(e_i-e_j)'\), with \(e_i\) and \(e_j\) unit
vectors\footnote{Unit vector $e_i$
has element $i$ equal to one and all other elements zero.}. Thus squared
distances can be expressed as
\(\|x_i-x_j\|^2=\text{tr}\ X'A_{ij}X=\text{tr}\ A_{ij}C\). In these new
variables the MDS problem is now to minimize \eqref{eq-csstress} over
all \(C\) with \(\text{rank}(C)\leq p\).

It is convenient to define \begin{subequations}
\begin{equation}
B:=\sum_{k=1}^m w_k\delta_k^2A_k,
\end{equation}
and 
\begin{equation}
V:=\sum_{k=1}^m w_ka_ka_k'
\end{equation}
\end{subequations} with \(a_k:=\text{vec}(A_k)\). Then
\(\sigma(c):=K-2b'c+c'Vc\), with \(c:=\text{vec}(C)\) and
\(b:=\text{vec}(B)\).

To start the quadratic majorization, use \(c=\tilde c+(c-\tilde c)\).
Then \begin{subequations}
\begin{equation}
\sigma(c)=\sigma(\tilde c)-2(c-\tilde c)'(b-V\tilde c)+(c-\tilde c)'V(c-\tilde c),
\end{equation} 
and thus 
\begin{equation}
\sigma(c)\leq
\sigma(\tilde c)+\lambda((c-\tilde c)-g)'((c-\tilde c)-g)-\lambda g'g,
\end{equation}
\end{subequations} with \(\lambda\) the largest eigenvalue of \(V\) and
\(g:=\lambda^{-1}(b-V\tilde c)\). In a majorization step we minimize
\((c-\overline c)'(c-\overline c),\) where
\(\overline{c}:=\tilde c+\lambda^{-1}(b-V\tilde c)\). {]}

Now \begin{equation}
V\tilde c=\sum_{k=1}^m w_ka_ka_k'\tilde c=\sum_{k=1}^m w_ka_k\text{tr}\ A_k\tilde C=\sum_{k=1}^m w_ka_kd_k^2(\tilde C),
\end{equation} and thus \begin{equation}
\text{vec}^{-1}(B-V\tilde c)=\sum_{k=1}^m w_k(\delta_k^2-d_k^2(\tilde C)A_k.
\label{eq-vecmin}
\end{equation} Equation \eqref{eq-vecmin} shows that in the majorization
stepm we need to minimize \(\text{tr}\ (\overline{C}-XX')^2\) with
\(\overline{C}:=\text{vec}^{-1}(\overline{c})\) over \(X\).

\sectionbreak

\section{Bound}\label{bound}

Computing \(\lambda\) is simplified by noting that the largest
eigenvalue of \(V\) is equal to the largest eigenvalue of
\(W^\frac12HW^\frac12\), where \(H\) has elements
\(h_{kl}=a_k'a_l^{\ }\).

The elements of \(H\) are all non-negative. Also \(h_{kl}\) is equal to
four if \(k=l\) and equal to one if \(A_k\) and \(A_l\) have one index
in common, otherwise it is zero. It follows that in the complete case,
with \(m=n(n-1)/2\), and in addition if there are unit weights,
\(\lambda=2n\). In the incomplete case, still with unit weights,
\(\lambda\leq 2n\).

If it is too expensive to calculate the largest eigenvalue, we can use
the bound \begin{equation}
\lambda=\max_x
\frac{x'W^\frac12HW^\frac12x}{x'x}=
\max_x\frac{x'W^\frac12HW^\frac12x}{x'Wx}\frac{x'Wx}{x'x}\leq 2n\max_k w_k.\label{eq-bound1}
\end{equation} This is a major improvement of the bound that is used,
either explicitly or implicitly, in De Leeuw
(\citeproc{ref-deleeuw_U_75b}{1975}) and De Leeuw, Groenen, and Pietersz
(\citeproc{ref-deleeuw_groenen_pietersz_E_16m}{2016}), which is
\begin{equation}
\lambda\leq\text{tr}\ WH=4\sum_{k=1}^m w_k.\label{eq-bound1}
\end{equation}

We can illustrate this with the colour data from Ekman
(\citeproc{ref-ekman_54}{1954}), using one minus the similarity as the
dissimilarity measurement and the square of the dissimlarity as weight.
The value of the eigenvalue bound is 19.21165, and smacofElegant() uses
154 iterations (with the default precision). If we use the bound from
\eqref{eq-bound1}, which is equal to \(2n=28\), then we need 219
iterations. And if we use \eqref{eq-bound2}, which is 245.324, we use
1579 iterations. Although the original \emph{ELEGANT} is indeed slow,
the bound modification makes it ten times as fast.

\sectionbreak

\section{Code}\label{code}

\begin{Shaded}
\begin{Highlighting}[]
\FunctionTok{library}\NormalTok{(RSpectra)}

\NormalTok{smacofElegant }\OtherTok{\textless{}{-}} \ControlFlowTok{function}\NormalTok{(theData,}
                          \AttributeTok{ndim =} \DecValTok{2}\NormalTok{,}
                          \AttributeTok{quick =} \ConstantTok{FALSE}\NormalTok{,}
                          \AttributeTok{itmax =} \DecValTok{1000}\NormalTok{,}
                          \AttributeTok{eps =} \FloatTok{1e{-}10}\NormalTok{,}
                          \AttributeTok{verbose =} \ConstantTok{TRUE}\NormalTok{) \{}
\NormalTok{  nobj }\OtherTok{\textless{}{-}}\NormalTok{ theData}\SpecialCharTok{$}\NormalTok{nobj}
\NormalTok{  ndat }\OtherTok{\textless{}{-}}\NormalTok{ theData}\SpecialCharTok{$}\NormalTok{ndat}
\NormalTok{  iind }\OtherTok{\textless{}{-}}\NormalTok{ theData}\SpecialCharTok{$}\NormalTok{iind}
\NormalTok{  jind }\OtherTok{\textless{}{-}}\NormalTok{ theData}\SpecialCharTok{$}\NormalTok{jind}
\NormalTok{  wght }\OtherTok{\textless{}{-}}\NormalTok{ theData}\SpecialCharTok{$}\NormalTok{weights}
\NormalTok{  wsum }\OtherTok{\textless{}{-}} \FunctionTok{sum}\NormalTok{(wght)}
\NormalTok{  delta }\OtherTok{\textless{}{-}}\NormalTok{ (theData}\SpecialCharTok{$}\NormalTok{delta)}\SpecialCharTok{\^{}}\DecValTok{2}
\NormalTok{  delta }\OtherTok{\textless{}{-}}\NormalTok{ delta }\SpecialCharTok{*} \FunctionTok{sqrt}\NormalTok{(wsum }\SpecialCharTok{/} \FunctionTok{sum}\NormalTok{(wght }\SpecialCharTok{*}\NormalTok{ delta}\SpecialCharTok{\^{}}\DecValTok{2}\NormalTok{))}
\NormalTok{  theData}\SpecialCharTok{$}\NormalTok{delta }\OtherTok{\textless{}{-}}\NormalTok{ delta}
\NormalTok{  cinit }\OtherTok{\textless{}{-}} \FunctionTok{smacofDoubleCenter}\NormalTok{(theData)}
\NormalTok{  lbda }\OtherTok{\textless{}{-}} \FunctionTok{smacofBound}\NormalTok{(theData, quick)}
\NormalTok{  evev }\OtherTok{\textless{}{-}} \FunctionTok{eigs\_sym}\NormalTok{(cinit, ndim)}
\NormalTok{  cinit }\OtherTok{\textless{}{-}} \FunctionTok{tcrossprod}\NormalTok{(evev}\SpecialCharTok{$}\NormalTok{vectors }\SpecialCharTok{\%*\%} \FunctionTok{diag}\NormalTok{(}\FunctionTok{sqrt}\NormalTok{(evev}\SpecialCharTok{$}\NormalTok{values)))}
\NormalTok{  dold }\OtherTok{\textless{}{-}} \FunctionTok{rep}\NormalTok{(}\DecValTok{0}\NormalTok{, ndat)}
  \ControlFlowTok{for}\NormalTok{ (k }\ControlFlowTok{in} \DecValTok{1}\SpecialCharTok{:}\NormalTok{ndat) \{}
\NormalTok{    i }\OtherTok{\textless{}{-}}\NormalTok{ iind[k]}
\NormalTok{    j }\OtherTok{\textless{}{-}}\NormalTok{ jind[k]}
\NormalTok{    dold[k] }\OtherTok{=}\NormalTok{ cinit[i, i] }\SpecialCharTok{+}\NormalTok{ cinit[j, j] }\SpecialCharTok{{-}} \DecValTok{2} \SpecialCharTok{*}\NormalTok{ cinit[i, j]}
\NormalTok{  \}}
\NormalTok{  sold }\OtherTok{\textless{}{-}} \FunctionTok{sum}\NormalTok{(wght }\SpecialCharTok{*}\NormalTok{ (delta }\SpecialCharTok{{-}}\NormalTok{ dold)}\SpecialCharTok{\^{}}\DecValTok{2}\NormalTok{) }\SpecialCharTok{/}\NormalTok{ wsum}
\NormalTok{  itel }\OtherTok{\textless{}{-}} \DecValTok{1}
\NormalTok{  cold }\OtherTok{\textless{}{-}}\NormalTok{ cinit}
  \ControlFlowTok{repeat}\NormalTok{ \{}
\NormalTok{    cmaj }\OtherTok{\textless{}{-}} \FunctionTok{matrix}\NormalTok{(}\DecValTok{0}\NormalTok{, nobj, nobj)}
    \ControlFlowTok{for}\NormalTok{ (k }\ControlFlowTok{in} \DecValTok{1}\SpecialCharTok{:}\NormalTok{ndat) \{}
\NormalTok{      i }\OtherTok{\textless{}{-}}\NormalTok{ iind[k]}
\NormalTok{      j }\OtherTok{\textless{}{-}}\NormalTok{ jind[k]}
\NormalTok{      cmaj[i, j] }\OtherTok{\textless{}{-}}\NormalTok{ cmaj[j, i] }\OtherTok{\textless{}{-}}\NormalTok{ wght[k] }\SpecialCharTok{*}\NormalTok{ (delta[k] }\SpecialCharTok{{-}}\NormalTok{ dold[k])}
\NormalTok{    \}}
\NormalTok{    cmaj }\OtherTok{\textless{}{-}} \SpecialCharTok{{-}}\NormalTok{cmaj}
    \FunctionTok{diag}\NormalTok{(cmaj) }\OtherTok{\textless{}{-}} \SpecialCharTok{{-}}\FunctionTok{rowSums}\NormalTok{(cmaj)}
\NormalTok{    cmaj }\OtherTok{\textless{}{-}}\NormalTok{ cold }\SpecialCharTok{+}\NormalTok{ cmaj }\SpecialCharTok{/}\NormalTok{ lbda}
\NormalTok{    evev }\OtherTok{\textless{}{-}} \FunctionTok{eigs\_sym}\NormalTok{(cmaj, ndim)}
\NormalTok{    xvev }\OtherTok{\textless{}{-}}\NormalTok{ evev}\SpecialCharTok{$}\NormalTok{vectors }\SpecialCharTok{\%*\%} \FunctionTok{diag}\NormalTok{(}\FunctionTok{sqrt}\NormalTok{(evev}\SpecialCharTok{$}\NormalTok{values))}
\NormalTok{    cnew }\OtherTok{\textless{}{-}} \FunctionTok{tcrossprod}\NormalTok{(xvev)}
\NormalTok{    dnew }\OtherTok{\textless{}{-}} \FunctionTok{rep}\NormalTok{(}\DecValTok{0}\NormalTok{, ndat)}
    \ControlFlowTok{for}\NormalTok{ (k }\ControlFlowTok{in} \DecValTok{1}\SpecialCharTok{:}\NormalTok{ndat) \{}
\NormalTok{      i }\OtherTok{\textless{}{-}}\NormalTok{ iind[k]}
\NormalTok{      j }\OtherTok{\textless{}{-}}\NormalTok{ jind[k]}
\NormalTok{      dnew[k] }\OtherTok{=}\NormalTok{ cnew[i, i] }\SpecialCharTok{+}\NormalTok{ cnew[j, j] }\SpecialCharTok{{-}} \DecValTok{2} \SpecialCharTok{*}\NormalTok{ cnew[i, j]}
\NormalTok{    \}}
\NormalTok{    snew }\OtherTok{\textless{}{-}} \FunctionTok{sum}\NormalTok{(wght }\SpecialCharTok{*}\NormalTok{ (delta }\SpecialCharTok{{-}}\NormalTok{ dnew)}\SpecialCharTok{\^{}}\DecValTok{2}\NormalTok{) }\SpecialCharTok{/}\NormalTok{ wsum}
    \ControlFlowTok{if}\NormalTok{ (verbose) \{}
      \FunctionTok{cat}\NormalTok{(}
        \StringTok{"itel "}\NormalTok{,}
        \FunctionTok{formatC}\NormalTok{(itel, }\AttributeTok{format =} \StringTok{"d"}\NormalTok{),}
        \StringTok{"sold "}\NormalTok{,}
        \FunctionTok{formatC}\NormalTok{(sold, }\AttributeTok{digits =} \DecValTok{10}\NormalTok{, }\AttributeTok{format =} \StringTok{"f"}\NormalTok{),}
        \StringTok{"snew "}\NormalTok{,}
        \FunctionTok{formatC}\NormalTok{(snew, }\AttributeTok{digits =} \DecValTok{10}\NormalTok{, }\AttributeTok{format =} \StringTok{"f"}\NormalTok{),}
        \StringTok{"}\SpecialCharTok{\textbackslash{}n}\StringTok{"}
\NormalTok{      )}
\NormalTok{    \}}
    \ControlFlowTok{if}\NormalTok{ ((itel }\SpecialCharTok{==}\NormalTok{ itmax) }\SpecialCharTok{||}\NormalTok{ ((sold }\SpecialCharTok{{-}}\NormalTok{ snew) }\SpecialCharTok{\textless{}}\NormalTok{ eps)) \{}
      \ControlFlowTok{break}
\NormalTok{    \}}
\NormalTok{    itel }\OtherTok{\textless{}{-}}\NormalTok{ itel }\SpecialCharTok{+} \DecValTok{1}
\NormalTok{    dold }\OtherTok{\textless{}{-}}\NormalTok{ dnew}
\NormalTok{    sold }\OtherTok{\textless{}{-}}\NormalTok{ snew}
\NormalTok{    cold }\OtherTok{\textless{}{-}}\NormalTok{ cnew}
\NormalTok{  \}}
  \FunctionTok{return}\NormalTok{(xvev)}
\NormalTok{\}}

\NormalTok{smacofDoubleCenter }\OtherTok{\textless{}{-}} \ControlFlowTok{function}\NormalTok{(theData) \{}
\NormalTok{  nobj }\OtherTok{\textless{}{-}}\NormalTok{ theData}\SpecialCharTok{$}\NormalTok{nobj}
\NormalTok{  ndat }\OtherTok{\textless{}{-}}\NormalTok{ theData}\SpecialCharTok{$}\NormalTok{ndat}
\NormalTok{  iind }\OtherTok{\textless{}{-}}\NormalTok{ theData}\SpecialCharTok{$}\NormalTok{iind}
\NormalTok{  jind }\OtherTok{\textless{}{-}}\NormalTok{ theData}\SpecialCharTok{$}\NormalTok{jind}
\NormalTok{  delta }\OtherTok{\textless{}{-}}\NormalTok{ (theData}\SpecialCharTok{$}\NormalTok{delta)}\SpecialCharTok{\^{}}\DecValTok{2}
\NormalTok{  cini }\OtherTok{\textless{}{-}} \FunctionTok{matrix}\NormalTok{(}\DecValTok{0}\NormalTok{, nobj, nobj)}
  \ControlFlowTok{for}\NormalTok{ (k }\ControlFlowTok{in} \DecValTok{1}\SpecialCharTok{:}\NormalTok{ndat) \{}
\NormalTok{    i }\OtherTok{\textless{}{-}}\NormalTok{ iind[k]}
\NormalTok{    j }\OtherTok{\textless{}{-}}\NormalTok{ jind[k]}
\NormalTok{    cini[i, j] }\OtherTok{\textless{}{-}}\NormalTok{ cini[j, i] }\OtherTok{\textless{}{-}}\NormalTok{ delta[k]}
\NormalTok{  \}}
\NormalTok{  rc }\OtherTok{\textless{}{-}} \FunctionTok{apply}\NormalTok{(cini, }\DecValTok{1}\NormalTok{, mean)}
\NormalTok{  rm }\OtherTok{\textless{}{-}} \FunctionTok{mean}\NormalTok{(cini)}
\NormalTok{  cini }\OtherTok{\textless{}{-}} \SpecialCharTok{{-}}\NormalTok{(cini }\SpecialCharTok{{-}} \FunctionTok{outer}\NormalTok{(rc, rc, }\StringTok{"+"}\NormalTok{) }\SpecialCharTok{+}\NormalTok{ rm) }\SpecialCharTok{/}  \DecValTok{2}
  \FunctionTok{return}\NormalTok{(cini)}
\NormalTok{\}}

\NormalTok{smacofBound }\OtherTok{\textless{}{-}} \ControlFlowTok{function}\NormalTok{(theData, }\AttributeTok{quick =} \ConstantTok{FALSE}\NormalTok{) \{}
\NormalTok{  ndat }\OtherTok{\textless{}{-}}\NormalTok{ theData}\SpecialCharTok{$}\NormalTok{ndat}
\NormalTok{  nobj }\OtherTok{\textless{}{-}}\NormalTok{ theData}\SpecialCharTok{$}\NormalTok{nobj}
\NormalTok{  iind }\OtherTok{\textless{}{-}}\NormalTok{ theData}\SpecialCharTok{$}\NormalTok{iind}
\NormalTok{  jind }\OtherTok{\textless{}{-}}\NormalTok{ theData}\SpecialCharTok{$}\NormalTok{jind}
\NormalTok{  wght }\OtherTok{\textless{}{-}}\NormalTok{ theData}\SpecialCharTok{$}\NormalTok{weights}
  \ControlFlowTok{if}\NormalTok{ (quick }\SpecialCharTok{==} \DecValTok{1}\NormalTok{) \{}
    \FunctionTok{return}\NormalTok{(}\DecValTok{2} \SpecialCharTok{*}\NormalTok{ nobj }\SpecialCharTok{*} \FunctionTok{max}\NormalTok{(wght))}
\NormalTok{  \}}
  \ControlFlowTok{if}\NormalTok{ (quick }\SpecialCharTok{==} \DecValTok{2}\NormalTok{) \{}
    \FunctionTok{return}\NormalTok{(}\DecValTok{4} \SpecialCharTok{*} \FunctionTok{sum}\NormalTok{(wght))}
\NormalTok{  \}}
\NormalTok{  asum }\OtherTok{\textless{}{-}} \DecValTok{4} \SpecialCharTok{*} \FunctionTok{diag}\NormalTok{(ndat)}
  \ControlFlowTok{for}\NormalTok{ (k }\ControlFlowTok{in} \DecValTok{1}\SpecialCharTok{:}\NormalTok{(ndat }\SpecialCharTok{{-}} \DecValTok{1}\NormalTok{)) \{}
\NormalTok{    kind }\OtherTok{\textless{}{-}} \FunctionTok{c}\NormalTok{(iind[k], jind[k])}
    \ControlFlowTok{for}\NormalTok{ (l }\ControlFlowTok{in}\NormalTok{ (k }\SpecialCharTok{+} \DecValTok{1}\NormalTok{)}\SpecialCharTok{:}\NormalTok{ndat) \{}
\NormalTok{      lind }\OtherTok{\textless{}{-}} \FunctionTok{c}\NormalTok{(iind[l], jind[l])}
\NormalTok{      asum[k, l] }\OtherTok{\textless{}{-}}\NormalTok{ asum[l, k] }\OtherTok{\textless{}{-}} \FunctionTok{sum}\NormalTok{(}\FunctionTok{outer}\NormalTok{(kind, lind, }\StringTok{"=="}\NormalTok{))}
\NormalTok{    \}}
\NormalTok{  \}}
\NormalTok{  asum }\OtherTok{\textless{}{-}}\NormalTok{ asum }\SpecialCharTok{*} \FunctionTok{sqrt}\NormalTok{(}\FunctionTok{outer}\NormalTok{(wght, wght))}
  \FunctionTok{return}\NormalTok{(}\FunctionTok{eigs\_sym}\NormalTok{(asum, }\DecValTok{1}\NormalTok{)}\SpecialCharTok{$}\NormalTok{values)}
\NormalTok{\}}
\end{Highlighting}
\end{Shaded}

\sectionbreak

\section*{References}\label{references}
\addcontentsline{toc}{section}{References}

\phantomsection\label{refs}
\begin{CSLReferences}{1}{0}
\bibitem[\citeproctext]{ref-browne_87}
Browne, M. W. 1987. {``{The Young-Householder Algorithm and the Least
Squares Multidimensional Scaling of Squared Distances}.''} \emph{Journal
of Classification} 4: 175--90.

\bibitem[\citeproctext]{ref-deleeuw_U_75b}
De Leeuw, J. 1975. {``{An Alternating Least Squares Approach to Squared
Distance Scaling}.''} Department of Data Theory FSW/RUL.

\bibitem[\citeproctext]{ref-deleeuw_C_94c}
---------. 1994. {``{Block Relaxation Algorithms in Statistics}.''} In
\emph{Information Systems and Data Analysis}, edited by H. H. Bock, W.
Lenski, and M. M. Richter, 308--24. Berlin: Springer Verlag.
\url{https://jansweb.netlify.app/publication/deleeuw-c-94-c/deleeuw-c-94-c.pdf}.

\bibitem[\citeproctext]{ref-deleeuw_E_16o}
---------. 2016. {``Convergence Rate of {ELEGANT} Algorithms.''} 2016.
\url{https://jansweb.netlify.app/publication/deleeuw-e-16-o/deleeuw-e-16-o.pdf}.

\bibitem[\citeproctext]{ref-deleeuw_groenen_pietersz_E_16m}
De Leeuw, J., P. Groenen, and R. Pietersz. 2016. {``An Alternating Least
Squares Approach to Squared Distance Scaling.''}
\url{https://jansweb.netlify.app/publication/deleeuw-groenen-pietersz-e-16-m/deleeuw-groenen-pietersz-e-16-m.pdf}.

\bibitem[\citeproctext]{ref-ekman_54}
Ekman, G. 1954. {``{Dimensions of Color Vision}.''} \emph{Journal of
Psychology} 38: 467--74.

\bibitem[\citeproctext]{ref-heiser_95}
Heiser, W. J. 1995. {``{Convergent Computing by Iterative Majorization:
Theory and Applications in Multidimensional Data Analysis}.''} In
\emph{Recent Advantages in Descriptive Multivariate Analysis}, edited by
W. J. Krzanowski, 157--89. Oxford: Clarendon Press.

\bibitem[\citeproctext]{ref-kearsley_tapia_trosset_94}
Kearsley, A. J., R. A. Tapia, and M. W. Trosset. 1994. {``{The Solution
of the Metric STRESS and SSTRESS Problems in Multidimensional Scaling
Using Newton's Method}.''} TR94-44. Houston, TX: {Department of
Computational and Applied Mathematics, Rice University}.
\url{https://apps.dtic.mil/sti/pdfs/ADA445621.pdf}.

\bibitem[\citeproctext]{ref-lange_16}
Lange, K. 2016. \emph{MM Optimization Algorithms}. SIAM.

\bibitem[\citeproctext]{ref-mishra_meyer_sepulchre_11}
Mishra, B., G. Meyer, and R. Sepulchre. 2011. {``Low-Rank Optimization
for Distance Matrix Completion.''} In \emph{50th IEEE Conference on
Decision and Control and European Control Conference, Orlando, FL, USA,
2011}, 4455--60. \url{https://arxiv.org/pdf/1304.6663}.

\bibitem[\citeproctext]{ref-takane_77}
Takane, Y. 1977. {``{On the Relations among Four Methods of
Multidimensional Scaling}.''} \emph{Behaviormetrika} 4: 29--42.

\bibitem[\citeproctext]{ref-takane_16}
---------. 2016. {``{My Early Interactions with Jan and Some of His Lost
Papers}.''} \emph{Journal of Statistical Software} 73 (7): 1--14.
\url{https://www.jstatsoft.org/article/view/v073i07}.

\bibitem[\citeproctext]{ref-takane_young_deleeuw_A_77}
Takane, Y., F. W. Young, and J. De Leeuw. 1977. {``Nonmetric Individual
Differences in Multidimensional Scaling: An Alternating Least Squares
Method with Optimal Scaling Features.''} \emph{Psychometrika} 42: 7--67.

\end{CSLReferences}




\end{document}
